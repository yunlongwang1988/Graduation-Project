\documentclass{article}
\usepackage[heading=true]{ctex} %heading=true中文版式生效
\usepackage{geometry}
\geometry{a4paper,left=2.5cm,right=2.5cm,top=2.5cm,bottom=2.5cm}%调整页边距
\usepackage{amsmath}
\usepackage{fancybox}
\usepackage{titlesec}%设置标题字号格
\usepackage{graphicx}
\graphicspath{{图片/}}
\usepackage{booktabs}
\usepackage{amsthm}
\usepackage{lipsum}
\usepackage{float}
\usepackage{fontspec}
\usepackage[dvipsnames,svgnames]{xcolor}
\usepackage[strict]{changepage} % 提供一个 adjustwidth 环境
\usepackage{cases}
\usepackage{tcolorbox}
\usepackage{framed} % 实现方框效果
\usepackage{newtxtext}
\usepackage{xcolor}
\usepackage{color}
\usepackage{caption}
\usepackage{subfigure}


%%%%%带底纹的文本
\definecolor{formalshade}{rgb}{0.95,0.95,1} % 文本框颜色
%% 注意行末需要把空格注释掉,不然画出来的方框会有空白竖线
\newenvironment{formal}{%
	\def\FrameCommand{%
		\hspace{1pt}%
		{\color{DarkBlue}\vrule width 2pt}%
		{\color{formalshade}\vrule width 4pt}%
		\colorbox{formalshade}%
	}%
	\MakeFramed{\advance\hsize-\width\FrameRestore}%
	\noindent\hspace{-4.55pt}% disable indenting first paragraph
	\begin{adjustwidth}{}{7pt}%
		\vspace{2pt}\vspace{2pt}%
	}	{%
		\vspace{2pt}\end{adjustwidth}\endMakeFramed%
}
%%%%%带底纹的文本

\title{\songti\zihao{3}毕业设计(论文)开题报告}
\author{王云龙}
\date{\today}

\renewcommand\thesection{\chinese{section}}
\renewcommand\thesubsection{\arabic{subsection}}
\renewcommand\thesubsubsection{\arabic{subsection}}	

\titleformat{\section}[block]{\raggedright\large\bfseries}{\thesection\,、}{1em}{}

\titleformat{\subsection}[block]{\raggedright\large\bfseries}{\arabic{section}.\arabic{subsection}}{1em}{}

\titleformat{\subsubsection}[block]{\raggedright\large\bfseries}{\arabic{section}.\arabic{subsection}.\arabic{subsubsection}}{1em}{}

\begin{document}
	\maketitle
	\fancypage{\fbox}{}
	\section{课题来源}
	
回流上古,追溯原始,从三百万年前非洲大陆的石器,到四千年前美索不达米亚的青铜,再到
一千四百年前小亚细亚的铁器,人类科技一直以一种小步慢跑的方式发展,直到十八世纪六十年代,
英格兰地区蒸汽动力的出现,标志着第一次工业革命的开始,它带领人类从农业文明进入到工业文
明,封建经济被资本主义经济取代,完成第一次工业革命的资本主义国家迅速开始全球扩张进行资
本积累,十九世纪七十年代,标志着以电气为代表的第二次工业革命的开始,资本主义飞速发展,世
界经济格局、资本主义世界市场形成,促进了工人运动和社会主义运动,第二次世界大战后,世界出
现新格局,以美国为首的北约和以苏联为首的华约之间的冷战,诞生了超乎人类想象力的科技,如
通用电器公司的Hardiman-动力装甲,SR-71 黑鸟式侦察机,XB-70 女武神式战略轰炸机,阿波罗计
划,地效飞行器,核潜艇等。


纵观人类科技发展历史,从最初的小步慢跑到井喷式发展,每一个扩展人类想象力科技的背后
都离不开工业机器的自动化生产,而当今世界正处于第四次工业革命的浪潮中,以人工智能,大数
据、机器人为代表的新技术正推动着工业生产进行新一轮的变革,工业自动化,信息化、智能化逐
渐成为未来制造业的发展趋势,传统制造业对于产业转型的需求日益迫切,在此背景下,国务院于
2015 年5 月提出《中国制造2025》十年行动纲领,目的就是为了加快推动新一代信息技术与传统
制造业融合发展,将智能制造作为主攻方向,着力发展智能装备和智能产品,推进生产过程智能化,
机器人凭借着强大的通用性,逐渐成为产业转型

	\section{国内外研究现状}
	\subsection{国内}
	\subsection{国外}
	
	\section{课题研究内容及实施方案}
	\subsection{研究内容}
	\subsection{实施方案}
	
	\section{参考文献}
	
	\section{进度安排}
	
	\section{指导老师审批意见}
\end{document}